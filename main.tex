\documentclass{article}
\usepackage[utf8]{inputenc}
\usepackage[spanish]{babel}
\usepackage{listings}
\usepackage{graphicx}
\graphicspath{ {images/} }
\usepackage{cite}
\usepackage{enumerate}

\begin{document}

\begin{titlepage}
    \begin{center}
        \vspace*{1cm}
            
        \Huge
        \textbf{Ideas Videojuego}
            
        \vspace{0.5cm}
        \LARGE
        Mountain of Dragons
            
        \vspace{1.5cm}
            
        \textbf{Juan Manuel Giraldo Botero}
            
        \vfill
            
        \vspace{0.8cm}
            
        \Large
        Despartamento de Ingeniería Electrónica y Telecomunicaciones\\
        Universidad de Antioquia\\
        Medellín\\
        Marzo de 2021
            
    \end{center}
\end{titlepage}

\newpage
\section{introduccion al juego}
mountain of dragons será un juego RPG basado en un mundo abierto donde el o los personajes exploraran el mapa para encontrar los objetos necesarios para las diferentes misiones y aventuras, siendo este un juego con historia y posibilidad de diferentes desenlaces, una variedad de enemigos y NPCs, las decisiones del jugador influirán en el transcurso de la historia y el entendimiento de esta.
\vspace{1cm}

\section{historia}
la historia tratará (por ahora) sobre el conflicto humano/dragón que ha llevado casi a la extinción de los dragones, en un tierra gobernada por seres aparentemente benevolentes, donde el personaje deberá cumplir con los ideales de estos o tomar su propio camino y aprender nuevas cosas, la historia tendrá explicaciones de sucesos anteriores que se desbloquearán según proceda el jugador, al igual que revelará quien es el verdadero enemigo.  
\vspace{1cm}

{\huge ideas técnicas}

\begin{enumerate}[1.]
    \item el jugador puede recorrer el mundo e interactuar con los NPCs para conseguir información, dinero o armas.
    \item tal vez se implemente un sistema para generar enemigos con habilidades y vida aleatorea, igual que sus movimientos.  .
    \item uno o dos minijuegos donde se podrán obtener ganancias.
    \item mejora de habilidades utilizando dinero.
    \item modo multijugador cooperativo .
    \item diferentes soundtracks para diferentes ambientaciones.
    \item menú para elegir personaje y nombre.
    \item tres finales diferentes.
    \item se podrá añadir diferentes objetos al equipamiento para cambiar los stats.
    \item opciones de sonido, y horario.

   (siendo honesto creo que esto será muy difícil pero haré mi mejor esfuerzo)

\end{enumerate}


\end{document}
